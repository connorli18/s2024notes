\documentclass[10pt]{extarticle} 
\usepackage{graphicx} % Required for inserting images
\usepackage{amsfonts}
\usepackage{enumitem}
\usepackage[hidelinks]{hyperref}
\usepackage{graphicx}
\usepackage{textcomp}
\usepackage{amsmath}
\usepackage{multicol}
\usepackage{mathabx}
\usepackage[bottom=2.5cm, right=1.5cm, left=1.5cm, top=2.5cm, headheight=16pt]{geometry}
\usepackage{amssymb}
\usepackage{amsthm}
\usepackage{amsmath}
\usepackage{physics}
\usepackage{cancel}
\usepackage{mathtools}
\usepackage{array}
\usepackage{tikz}
\def\checkmark{\tikz\fill[scale=0.4](0,.35) -- (.25,0) -- (1,.7) -- (.25,.15) -- cycle;} 
\makeatletter
\newcases{crcases}{\quad}{%
  \hfil$\m@th\displaystyle{##}$\hfil}{\hfil$\m@th\displaystyle{##}$}{\lbrace}{.}
\makeatother
\usepackage{mdframed}
\usepackage{tikzlings}
\usepackage{tikzducks}
\usepackage{MnSymbol}
\usepackage{animate}
\usepackage{physics}
\usepackage{url} % Import url package
\usepackage{listings}
\usepackage{fancyhdr}

% Setting the appearance of the listings
\lstset{
    basicstyle=\ttfamily\small,
    columns=fullflexible,
    breaklines=true,
    language=Python, % specify the programming language
    frame=single, % adds a frame around the code
    showstringspaces=false % do not replace spaces in strings with a visual marker
}



\pagestyle{fancy}
\fancyhf{}
\fancyhead[R]{\thepage} % Left header
\renewcommand{\headrulewidth}{0pt} % Remove header line


\title{[IoT Firewall] Research Notes}
\author{Connor Li, csl2192}
\date{Spring 2024}

\newmdenv{boxedsection}

\begin{document}

\maketitle

\section*{Notes}
How to set up the network topology.
\begin{lstlisting}[language=Python]
    # start network topology
    make clean
    make

    # in mininet
    xterm h1 h2 h3

    # new terminal
    sudo python3 controller.py

    # new terminal
    # will be used for blocking
\end{lstlisting}
Let's show how regular packets get sent through.
\begin{lstlisting}
    # in h2, h3
    sudo python3 recieve.py

    # in h1
    sudo python3 send.py h2/h3 'I love P4'

    # You should see in controller terminal that packets are being sent
    # You should see that h2, h3 are printing out packets
\end{lstlisting}
Now, let's show that large packets are dropped.
\begin{lstlisting}
    # in h2
    sudo python3 recieve.py

    # in h1
    sudo python3 large-send.py h2

    # You should see that the controller is registering sending packets of size > 1000
    # You should see that h2, h3 are not printing out packets because the switch drops them
\end{lstlisting}
Now, let's show that we can write table rules to drop packets.
\begin{lstlisting}
    # in h2, h3
    sudo python3 recieve.py

    # in h1
    sudo python3 send.py h2/h3 'I love P4'

    # You should see that the controller is registering sending packets
    # You should see that the packets are going through

    # Now, in a new terminal, run the following
    sudo python3 block.py h2

    # Now, in h1, try sending packets
    sudo python3 send.py h2/h3 'I love P4'

    # You should see that the controller is registering sending packets
    # You should see that the packets are not going through to h2, but still going through to h3
\end{lstlisting}
Now, restart the controller. Let's show the packet-rate blocking feature.
\begin{lstlisting}
    #new terminal, refresh the controller and the mininet
    make clean
    make
    xterm h1 h2 h3
    sudo python3 controller.py
    
    # in h2
    sudo python3 recieve.py

    # in h1
    sudo python3 spam-send.py h2 --duration 10 'i love p4'

    # You should see that the controller is registering sending packets
    # You should see in the controller temrinal that h1 disconnects!
\end{lstlisting}

\end{document}