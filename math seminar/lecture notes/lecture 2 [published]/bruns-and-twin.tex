\documentclass[8pt]{extarticle} 
\usepackage{graphicx} % Required for inserting images
\usepackage{amsfonts}
\usepackage{enumitem}
\usepackage[hidelinks]{hyperref}
\usepackage{graphicx}
\usepackage{textcomp}
\usepackage{amsmath}
\usepackage{multicol}
\usepackage{mathabx}
\usepackage[bottom=0.5cm, right=1.5cm, left=1.5cm, top=1.5cm, headheight=16pt]{geometry}
\usepackage{amssymb}
\usepackage{amsthm}
\usepackage{amsmath}
\usepackage{physics}
\usepackage{cancel}
\usepackage{mathtools}
\usepackage{array}
\usepackage{tikz}
\def\checkmark{\tikz\fill[scale=0.4](0,.35) -- (.25,0) -- (1,.7) -- (.25,.15) -- cycle;} 
\makeatletter
\newcases{crcases}{\quad}{%
  \hfil$\m@th\displaystyle{##}$\hfil}{\hfil$\m@th\displaystyle{##}$}{\lbrace}{.}
\makeatother
\usepackage{mdframed}
\usepackage{tikzlings}
\usepackage{tikzducks}
\usepackage{MnSymbol}
\usepackage{animate}
\usepackage{physics}



\title{[Math Seminar] Brun's Combinatorial Sieve \& Twin Primes}
\author{Connor Li, csl2192}
\date{March 20, 2024}

\newmdenv{boxedsection}

\begin{document}

\maketitle

\section*{Important Resources}
\begin{itemize}
    \item \textbf{Email:} \href{mailto:csl2192@columbia.edu}{csl2192@columbia.edu}
    \item \textbf{Link to Class Webpage:} \href{https://www.math.columbia.edu/~avizeff/additive/index.html}{Click Here}
    \item \textbf{[Avi] Seminar Part 2:} \href{https://www.math.columbia.edu/~avizeff/additive/talk_9.5.pdf}{Click Here}
    \item \textbf{[Johnny] Introduction to Sieves:} \href{https://www.math.columbia.edu/~avizeff/additive/talk_10.pdf}{Click Here}
    \item \textbf{Reference Material 1:} \href{http://www.alefenu.com/libri/nathansonbases.pdf}{Additive Number Theory (Section 6.4)}
    \item \textbf{Reference Material 2:} \href{https://pages.cs.wisc.edu/~cdx/Sieve.pdf}{Sieve Methods (Section 2.1)}
\end{itemize}

\tableofcontents   


\pagebreak
\section{Review of Sieves}
\subsection{Introduction}
Last time, Johnny introduced the concept of sieves. Specifically, sieve theory is a set of general techniques in number theory designed to count/estimate the size of sifted sets of integers.
The prime example of a sieve is the Sieve of Eratosthenes, which is a simple, ancient algorithm for finding all prime numbers up to a specified integer.\\
\\
And although we won't cover the concepts of sieves in depth (since Johnny already did a great job last lecture), we will be using the Sieve of Eratosthenes as a starting point to understand Brun's Combinatorial Sieve.
\subsection{Conceptual Review}
The way the counting argument works for inclusion exclusion is we start with some set $\mathcal{A}$ consisting of all the positive integers up to $x$. Then, we begin by removing and readding sets to determine the size of $A$ once we've removed all the non-desired elements. In this case, you can image that the desired elements are the prime numbers and we being by removing all the multiples of $2$. Next, we remove the multiples of $3$, but we have to readd the overlap (a number lie $6$). Otherwise, the number $6$ would be removed from the set twice and would cause an inaccuracy in our true estimation of the size of the prime numbers up to $x$, also denoted as the set $\mathcal{A}'$ with all multiples removed.\\
\\
If that was too difficult to follow, don't worry! Essentially, the sieve of Eratosthenes in its ``full glory'' says, under certain technical assumptions, that the sum (which we use to approximate the size of some desired set) is asymptotic to some very mathematically-nice functions plus some error term.
\subsection{Connection to Brun's Sieve}
So, you might be wondering, how does it all connect? Well, Brun's pure sieve improves on that of Ertatosthenes by using a more sophisticated inclusion-exclusion argument to count the number of twin primes up to $x$. Specifically,
\begin{itemize}
    \item It relaxes the assumptions in the sieve of Eratosthenes a little bit. The Sieve of Eratosthenes is used to explicitly enumerate prime numbers within a range, giving an exact count. Brun's sieve, in contrast, is more about estimation. It provides an upper bound on the count of certain prime configurations, like twin primes, without necessarily listing them. This relaxation allows for analytical techniques to estimate the distribution of primes without exhaustive computation.
    \item The asymptotic approximation is done with a much better error term because Brun's sieve specifically targets the distribution of primes that have a specific gap between them, like twin primes (primes that are two units apart). The classical sieve doesn't inherently provide information on prime gaps or configurations; it's primarily a filtering algorithm
\end{itemize}
\pagebreak
\section{Brun's Sieve}
\subsection{Introduction}
Brun's sieve is a combinatorial sieve that is used to estimate the number of twin primes up to a given number $x$. A twine prime is a pair of prime numbers that differ by $2$. For example, $(3,5)$ and $(11,13)$ are twin primes. 
This technique to count twin primes was first introduced by the Norwegian mathematician Viggo Brun in 1919 as an extension of the sieve of Eratosthenes.\\
\\
An important result related to this discussion is the twin prime conjecture, which states that there are inifinitely many twin primes. Equivalently, we can also say that there are infinitely many integers such that $k(k+2)$ has exactly two prime factors. Later, Jinoo will also talk about a related conjecture called Goldbach's conjecture. \\
\\
Remember, the goal of this sieve is to show that the twin primes are sparse compared to the prime numbers. From a result in last lecture, we know that the sum of reciprocals of prime numbers diverge.
$$
\sum_{p \leq x} \frac{1}{p} = \log \log x + B + O\left(\frac{1}{\log x}\right) \implies \lim_{x\rightarrow\infty} \sum_{p \leq x} \frac{1}{p} = \infty
$$
However, since twin primes are far less frequent than prime numbers, we can show that the sum of reciprocals of twin primes converges. This is the main idea behind Brun's sieve.
$$
\lim_{x\rightarrow \infty} \sum_{\substack{p_1, p_2 \leq x \\ p_2 = p_1 + 2}} \frac{1}{p_1} + \frac{1}{p_2} < \infty
$$
\subsection{Inclusion-Exclusion}
Like I mentioned before, Brun's sieve is based on the logic of inclusion-exclusion. As a result, we will prove a theorem to formalize that method of choosing. The combinatiorial inequality that we will show is the simplest form of the Brun sieve.
\begin{boxedsection}
\textbf{Theorem:} If $\ell \geq 1$ and $0 \leq m \leq \ell$, then
$$
\sum_{k=0}^m (-1)^k {\ell \choose k}  = (-1)^m {\ell - 1 \choose m}
$$
\textbf{Proof:} We will prove this by induction. The base case is when $m = 0,1,2$. If $m = 0$, the case is trivial. If $m = 1$, then we have 
$$
\sum_{k=0}^1 (-1)^k {\ell \choose k} = 1 - \ell = (-1)^1 {\ell - 1 \choose 1}
$$
Similarly, for $m = 2$, we have the following.
$$
\sum_{k=0}^2 (-1)^k {\ell \choose k} = 1 - \ell + {\ell \choose 2} = \frac{(\ell-1)(\ell -2)}{2} = (-1)^2 {\ell - 1 \choose 2}
$$
Now, use induction on $m$. Assume the equation holds true for $m-1$, then we have
\begin{align*}
    \sum_{k=0}^m (-1)^k {\ell \choose k} &= \sum_{k=0}^{m-1} (-1)^k {\ell \choose k} + (-1)^m {\ell \choose m}\\
    &= (-1)^{m-1} {\ell - 1 \choose m-1} + (-1)^m {\ell \choose m}\\
    &= (-1)^m \left({\ell \choose m} - {\ell - 1 \choose m-1}\right)\\
    &= (-1)^m {\ell - 1 \choose m}
\end{align*}
Thus, we have completed the proof.
\end{boxedsection}
\subsection{The Brun Sieve}
Now, we will use the above theorem to introduce the Brun sieve as a combinatorial argument.
\begin{boxedsection}
\textbf{Theorem:} Let $X$ be a nonempty, finite set where $|X| = N$. Now, let $P_1, \dots, P_r$ be $r$ distinct properties that elements of set $X$ might have. Let $N_0$ denote the number of elements of $X$ that have none of these properties. \\
\\
For any subset $I = \{i_1, \dots, i_k\}$ of $\{1,2,\dots, r\}$ let $N(I)$ denote the number of elements of $X$ that have each of the properties $P_{i_1}, P_{i_2}, \dots P_{i_k}$. And if $I = \varnothing$, then we assume that $N(\varnothing) = |X| = N$. If $m$ is an even integer, then
$$
N_0 \leq \sum_{k=0}^m (-1)^k \sum_{|I| = k} N(I)
$$
If $m$ is an odd integer, then 
$$
N_0 \geq \sum_{k=0}^m (-1)^k \sum_{|I| = k} N(I)
$$
\textbf{Note:} The intuition behind this formula os to provide a bound on the number of elements of $X$. If you take the properties $P_r$ to be ``being divisible by some prime,'' then we are able to construct bounds for a set $N_0$ that only consists of the prime numbers. Essentially, you should view this as a generalized method to pick and choose certain primes to construct your own set.
\end{boxedsection}
\pagebreak
\begin{boxedsection}
\textbf{Proof:} The inequalities in the theorem count the elements of $X$ as explained above. To prove the theorem, let's focus on the contribution from each individual element $x \in X$ to these inequalities.\\
\\
Assume that $x$ has exactly $\ell$ properties $P_i$. If $\ell = 0$, then $x$ is counted once in $N_0$ and once in $N(\varnothing)$, but it is not counted in $N(I)$ if $I$ is nonempty.  If $\ell \geq 1$, then $x$ is not counted in $N_0$. But, since $\ell \geq 1$, we can renumber the properties $P_1, \dots, P_\ell$ such that $x$ has those properties. Now, define some set of indexes of attributes $I \subset \{1,\dots,r\}$.\\
\\
If $\exists \; i \in I$ such that $i > \ell$, then $x$ is not counted in $N(I)$ because it is missing one of the properties of $I$. However, if $I \subset \{1,\dots,\ell\}$, then $x$ is counted once in $N(I)$ since it contains all the properties. This means that for each $k = 0,1,\dots,\ell$, there are exactly ${\ell \choose k}$ subsets $I$ such that $|I| = k$. If $m \geq \ell$, the element $x$ contributes 
$$
\sum_{k=0}^\ell (-1)^k {\ell \choose k} = 0
$$
This is because of a property of the binomial theorem that allows you to express the left hand side with the summation as $0^n$ for some $n \in \mathbb{N}$, which evaluates to $0$. However, if $m < \ell$, the binomial terms don't all cancel out and the element $x$ contributes
$$
\sum_{k=0}^m (-1)^k {\ell \choose k}
$$
It's important to note that if $\ell$ is even, then the contribution is positive and we have an upper-bound. But if $\ell$ is odd, then the contribution is negative and we have a lower-bound as stated in the theorem.
\end{boxedsection}
You might be asking yourself, why is this count not exact? Well, it's a truncated sum, which means that it's incomplete. Since you aren't summing over all properties, you lack the adjustment needed for the $m+1$ property. Thus, we have a bound instead of an exact count, and if you use the Inclusion Exclusion Theorem in 2.2, we can show the negative/positive bound.
\section{Twin Primes}
\subsection{Introduction}
The goal of this next section is to show an asymptotic bound for the twin primes, which are pairs of prime numbers that differ by $2$. Utlimately, we will use the sieve above to show something meaningful, but we need a couple of lemmas first.
\subsection{Precursors to Theorem}
Although I think the proofs are valuable, they require a lot of technical details that are not necessary for the understanding of the main idea. Thus, I will only state the lemmas and their results.
\begin{boxedsection}
    \textbf{Theorem:} (Lemma $1$) For $x \geq 1$ and for any congruence class $a \text{ mod } m$, the number of positive integers not exceeding $x$ that are congruent to $a \text{ mod } m$ is $\frac{x}{m} + \theta$ where $|\theta| < 1$. \\
    \\
    \textbf{Proof:} Although the actual proof requires skills from Algebra, the intuition is pretty simple. If you were to divide up the integers from $1$ to $x$ into $m$ groups based on congruence classes, then each class would have $\frac{x}{m}$ elements. However, if $x$ is not divisible by $m$, then there will be a remainder $\theta$ that is less than $1$. This is the main idea behind the proof.
\end{boxedsection}
The second proof is going to tell us a better quantifiable bound on the set $N(I)$ from the Brun sieve. However, for the specific case of twin primes, we will define the set $N(i_1, \dots, i_k)$ such that $N$ consists of the integers divisble by some set of primes $p_{i_1}, \dots, p_{i_k}$.
\begin{boxedsection}
    \textbf{Theorem:} (Lemma $2$) Let $x \geq 1$, and let $p_{i_1}, \dots p_{i_k}$ be distinct odd primes. Let $N(i_1, \dots, i_k)$ denote the number of positive integers $n \leq x$ such that
    $$
    n(n+2) \equiv 0 \text{ mod } p_{i_1} \cdots p_{i_k}
    $$
    Then, for $|\theta| < 1$,
    $$
    N(i_1, \dots, i_k) = \frac{2^k x}{p_{i_1} \cdots \;p_{i_k}} + 2^k\theta
    $$
    \textbf{Proof:} For the proof, we use the above lemma but extend it to the case of numbers constructed like $n(n+2)$. Although the proof isn't complicated, I don't think it's worth mentioning. If you're interested, please reference the textbook Section 6.4 Lemma 6.8.
\end{boxedsection}
\subsection{Brun's Theorem}
\begin{boxedsection}
    \textbf{Theorem:} Let $\pi_2(x)$ denote the number of primes $p$ not exceeding $x$ such that $p+2$ is also prime. Then,
    $$
    \pi_2(x) \ll \frac{x (\log \log x)^2}{(\log x)^2}
    $$
    \textbf{Proof: (Part 1)} Let $5 \leq y < x$ and let $r = \pi(y) - 1$ represent the number of odd primes not exceeding $y$, which we will also enumerate as $p_1, \dots, p_r$. Now, let $\pi_2(y,x)$ denote the 
    number of primes $p$ such that $y < p \leq x$ and $p+2$ is also prime. If $y < n \leq x$ and both $n, n+2$ are prime numbers, then $n > p_i$ for all $i \in [1,r]$ and for all $p_i$,
    $$
    n(n+2) \nequiv 0 \text{ mod } p_i
    $$
    Let $N_0(y,x)$ denotes the number of positive integers $n \leq x$ such that $n(n+2) \nequiv 0 \text{ mod } p_i$ for all $i \in [1,r]$. Then, we have a very simple upper-bound
    $$
    \pi_2(y) \leq y + \pi_2(y,x) \leq y + N_0(y,x)
    $$
    Now, we can simply use Brun's sieve and the lemmas above to find an upper-bound for $N_0(y,x)$.
\end{boxedsection}



\begin{boxedsection}
    \textbf{Proof: (Part 2)} Define $X$ to be the set of positive integers not exceeding $x$. 
    For each odd prime $p_i \leq y$, we let $P_i$ be the property that $n(n+2)$ is divisible by $p_i$. This means for any subset $I = \{i_1, \dots, i_k\}$ of $\{1,2,\dots,r\}$, we have $N(I)$ denote the number of elements of $X$ that have each of the properties $P_{i_1}, P_{i_2}, \dots, P_{i_k}$. In other words, $N(I)$ denotes the number of integers $n \in X$ 
    such that $n(n+2)$ is divisble by each of the primes $p_{i_1}, \dots, p_{i_k}$ or that $n(n+2)$ is divisible by $p_{i_1}\cdots\; p_{i_k}$.\\
    \\
    Using our Lemma $2$ from above, we have that 
    $$
    N(I) = N(i_1, \dots, i_k) = \frac{2^k x}{p_{i_1} \cdots\; p_{i_k}} + 2^k\theta
    $$
    Now, let $m$ be an even integer such that $1 \leq m \leq r$. Using our inequality from the Inclusion-Exclusion section, we have the following result.
    \begin{align*}
        N_0(y,x) &\leq \sum_{k=0}^m (-1)^k \sum_{|I| = k} N(I)\\
        &\leq \sum_{k=0}^m (-1)^k \sum_{\{i_1, \dots, i_k\} \subset \{1,\dots,r\}} \left(\frac{2^k x}{p_{i_1} \cdots \;p_{i_k}} + 2^k\theta\right)\\
        &\leq x \sum_{k=0}^m \sum_{\{i_1, \dots, i_k\} \subset \{1,\dots,r\}} \frac{(-2)^k}{p_{i_1} \cdots \;p_{i_k}} + \sum_{k=0}^m (-1)^k {r \choose k} \mathcal{O}(2^k)\\
        &\leq \underbrace{x \sum_{k=0}^r \sum_{\{i_1, \dots, i_k\} \subset \{1,\dots,r\}} \frac{(-2)^k}{p_{i_1} \cdots \;p_{i_k}}}_{S_1} - \underbrace{x \sum_{k=m+1}^r \sum_{\{i_1, \dots, i_k\} \subset \{1,\dots,r\}} \frac{(-2)^k}{p_{i_1} \cdots \;p_{i_k}}}_{S_2} + \underbrace{\mathcal{O}\left(\sum_{k=0}^m {r \choose k} 2^k\right)}_{S_3}
    \end{align*}
    As you can see, the above expression is composed of $3$ parts, which we will evaluate separately. Using Merten's Formula, we can show the following bound for $S_1$.
    \begin{align*}
        S_1 = x \sum_{k=0}^r \sum_{\{i_1, \dots, i_k\} \subset \{1,\dots,r\}} \frac{(-2)^k}{p_{i_1} \cdots \;p_{i_k}} &= x \prod_{2 < p \leq y} \left(1- \frac{2}{p}\right)\\
        &< x \prod_{2 < p \leq y} \left(1- \frac{1}{p}\right)^2\\
        &\ll \frac{x}{(\log y)^2}
    \end{align*}
    Now, consider $S_2$. Let $s_k(x_1, \dots, x_r)$ be the elementary symmetric polynomial of degree $k$ in $r$ variables. For any nonnegative real numbers $x_1, \dots, x_r$, we have
    \begin{align*}
        s_k(x_1, \dots, x_r) &= \sum_{\{i_1,\dots,i_k\} \subset \{1,\dots,r\}} x_{i_1} \cdots x_{i_k}\\
        &\leq \frac{(x_1 + \cdots + x_r)^k}{k!}\\
        &= \frac{(s_1(x_1, \dots, x_r))^k}{k!}\\
        &< \left(\frac{e}{k}\right)^k (s_1(x_1, \dots, x_r))^k
    \end{align*}
    Using this fact, we have the following bound for $S_2$.
    \begin{align*}
        |S_2| = \left|x \sum_{k=m+1}^r \sum_{\{i_1, \dots, i_k\} \subset \{1,\dots,r\}} \frac{(-2)^k}{p_{i_1} \cdots \;p_{i_k}}\right| &\leq x \sum_{k=m+1}^r \sum_{\{i_1, \dots, i_k\} \subset \{1,\dots,r\}} \frac{2^k}{p_{i_1} \cdots \;p_{i_k}}\\
        &\leq x \sum_{k=m+1}^r \sum_{\{i_1, \dots, i_k\} \subset \{1,\dots,r\}} \left(\frac{2}{p_{i_1}}\right) \cdots \left(\frac{2}{p_{i_k}}\right)\\
        &= x \sum_{k=m+1}^r s_k\left(\frac{2}{p_1}, \dots, \frac{2}{p_r}\right)\\
        &< x \sum_{k=m+1}^r \left(\frac{e}{k}\right)^k \left(s_1\left(\frac{2}{p_1}, \dots, \frac{2}{p_r}\right)\right)^k\\
        &= x \sum_{k=m+1}^r \left(\frac{e}{k}\right)^k \left(\frac{2}{p_1} + \cdots + \frac{2}{p_r}\right)^k\\
        &< x \sum_{k=m+1}^r \left(\frac{2e}{m}\right)^k \left(\sum_{p \leq y} \frac{1}{p}\right)^k\\
        &< x \sum_{k=m+1}^r \left(\frac{c \log \log y}{m}\right)^k\\
        &\leq \underbrace{x \sum_{k=m+1}^\infty \frac{1}{2^k} < \frac{x}{2^m}}_{m \;>\; 2c\log \log y}
    \end{align*}
\end{boxedsection}
\pagebreak
\begin{boxedsection}
    \textbf{Proof: (Part 3)} Finally, we have the following bound for $S_3$. Since $r$ is the number of odd primes less than or equal to $y$, it follows that $2r \leq y$, so we can bound $S_3$ as follows.
    \begin{align*}
        S_3 = \sum_{k=0}^m {r \choose k} 2^k &\leq \sum_{k=0}^m r^k 2^k \ll (2r)^m \leq y^m
    \end{align*}
    Thus, if we backtrack to our original estimate for $\pi_2(y)$, we have the following for any $5 \leq y < x$ and $m > 2c \log \log y$ where $c$ is fixed.
    $$
    \pi_2(y) \leq y + N_0(y,x) \leq y + S_1 - S_2 + S_3 \ll \frac{x}{(\log y)^2} + \frac{x}{2^m} + y^m
    $$
    Let $c' = \max\{2c, \frac{1}{\log 2}\}$ and define $y$ as follows.
    $$
    y = \exp\left(\exp\left(\frac{\log x}{3c'\log \log x}\right)\right) \quad \quad \quad m = 2c' \log \log x
    $$
    This definition for $y$ satisfies the necessary bounding conditions for $x$ sufficiently large. And since $\log y = \frac{\log x}{3c' \log \log x}$, we can bound $S_1$.
    $$
    \frac{x}{(\log y)^2} \ll \frac{x (\log \log x)^2}{(\log x)^2}
    $$
    For $S_2$, use our same definitions of $y,m$ to show the following.
    $$
    \frac{x}{2^m} < \frac{4x}{2^{2c' \log \log x}} \leq \frac{4x}{(\log x)^{2c' \log 2}} \ll \frac{4x}{(\log x)^2}
    $$
    For $S_3$, do the same and we get the following.
    $$
    y^m \leq y^{2c' \log \log x} = \exp\left(\frac{2c' \log \log x \log x}{3c' \log \log x}\right) = x^{\frac{2}{3}}
    $$
    Finally, if we combine these three estimates, we get the following result.
    $$
    \pi_2(x) \ll  \frac{x (\log \log x)^2}{(\log x)^2} +  \frac{4x}{(\log x)^2} + x^{\frac{2}{3}} \ll \frac{x (\log \log x)^2}{(\log x)^2}
    $$
    This completes our proof.
\end{boxedsection}



\subsection{Twin Primes}
This is the famous conjecture about twin primes and their reciprocals converging.
\begin{boxedsection}
    \textbf{Theorem:} Let $p_1, p_2, \dots$ be a sequence of prime numbers $p$ such that $p+2$ is also prime. Then,
    \begin{align*}
        \sum_{n=1}^\infty \left(\frac{1}{p_n} + \frac{1}{p_n+2}\right) &= \left(\frac{1}{3} + \frac{1}{5}\right) + \left(\frac{1}{5} + \frac{1}{7}\right) + \left(\frac{1}{11} + \frac{1}{13}\right) + \cdots\\
        &< \infty
    \end{align*}
    \textbf{Proof:} From the above Brun's Theorem, we know the following for $x \geq 2$.
    $$
    \pi_2(x) \ll \frac{x (\log \log x)^2}{(\log x)^2} \ll \frac{x}{(\log x)^{\frac{3}{2}}}
    $$
    Thus, we also have that for some $n = \pi_2(p_n)$.
    $$
    n = \pi_2(p_n) \ll \frac{p_n}{(\log p_n)^{\frac{3}{2}}} \leq \frac{p_n}{(\log n)^\frac{3}{2}}
    $$
    If we rearrange the terms on both sides, we can get the following inequality.
    $$
    n (\log n)^{\frac{3}{2}} \ll p_n \implies \frac{1}{p_n} \ll \frac{1}{n (\log n)^{\frac{3}{2}}}
    $$
    Thus, we can prove that the series converges.
    $$
    \sum_{n=1}^\infty \frac{1}{p_n} = \frac{1}{3} + \sum_{n=2}^\infty \frac{1}{p_n} \ll \frac{1}{3} + \sum_{n=2}^\infty \frac{1}{n (\log n)^{\frac{3}{2}}} < \infty
    $$
\end{boxedsection}
\section{Conclusion}
In conclusion, we have shown that the twin primes are sparse compared to the prime numbers. This is a result of the Brun sieve, which is a combinatorial sieve that is used to estimate the number of twin primes up to a given number $x$. The sieve is based on the logic of inclusion-exclusion, and we used the sieve to show that the sum of reciprocals of twin primes converges. This is a very important result in number theory, and it is a testament to the power of sieve methods in number theory.\\
\\
Please let me know if you have any questions. For more reference material, there are a lot of online resources that can help you understand the sieve methods in number theory. 
\end{document}
